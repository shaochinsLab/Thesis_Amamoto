
\documentclass[dvipdfmx]{beamer}
\AtBeginShipoutFirst{\special{pdf:tounicode EUC-UCS2}}

\usetheme{Madrid}
\usecolortheme{default}
\setbeamerfont{frametitle}{size=\large,series=\bfseries}
\setbeamertemplate{navigation symbols}{}
\setbeamertemplate{footline}[frame number]
\setbeamertemplate{items}[circle]
\setbeamertemplate{section in toc}[circle]
\setbeamertemplate{blocks}[rounded][shadow=true]

\definecolor{character}{RGB}{0,0,0} %文字
\definecolor{background}{RGB}{245,245,245} %背景
\setbeamercolor{normal text}{bg=background, fg=character}

\definecolor{mathFormula}{RGB}{53,121,53} %数式
\setbeamercolor{math text}{fg=mathFormula}

\definecolor{alertColor}{RGB}{255,70,70} %alert
\setbeamercolor{alerted text}{fg=alertColor}

\definecolor{blockAlertColor}{RGB}{0,0,0} %alertblock内の文字
\setbeamercolor{block body alerted}{fg=blockAlertColor}

\definecolor{headColor}{RGB}{52,38,89} %見出しカラー
\setbeamercolor{structure}{fg=headColor}
\setbeamercolor{subsection in toc}{fg=headColor}

\usepackage[absolute,overlay]{textpos}
\setbeamercovered{dynamic}
% \usepackage[colorgrid,gridunit=pt,texcoord]{eso-pic}


% \mathversion{bold}
% \usefonttheme[onlymath]{serif}
% 文字フォント設定
% \setbeamertemplate{blocks}[rounded] % Blockの影を消す
% \mathversion{bold} %数式を太字に
\usefonttheme{professionalfonts}% 数式用フォント
\setbeamertemplate{items}[default] % itemize 変更
\renewcommand{\kanjifamilydefault}{\gtdefault}  % 日本語をゴシック体に
\renewcommand{\familydefault}{\sfdefault} %英字をサンセリフに
\newcommand{\argmin}{\mathop{\rm arg~min}\limits}
%\renewcommand{\familydefault}{\rmdefault} %英字をローマンに
%\setbeamertemplate{section in toc}[square] %目次を球体から四角へ
% \newtheorem{remark}[theorem]{Remark}
% \newtheorem{observation}[theorem]{Observation}


\title{並列機械モデルにおける\\最大実行開始待ち時間最小化問題の計算論的評価}
\author{天本 祐希}
\institute{青山学院大学 宋研究室}
\date{\today}
\begin{document}

%%%%%%%%%%%%%%%%%%%%%%%%%%%%%%%%%%%%%%%%%%%%%%%%%%%%%%%%%%%%%%%%%%%
\begin{frame}
  \titlepage
\end{frame}

\begin{frame}{目次}
  \tableofcontents
\end{frame}
%%%%%%%%%%%%%%%%%%%%%%%%%%%%%%%%%%%%%%%%%%%%%%%%%%%%%%%%%%%%%%%%%%%
\section{研究背景}
\begin{frame}{研究背景:Web サービスにおける処理の流れと問題}
  \begin{figure}[h]
    \centering
    \includegraphics<1>[width=12cm]{figure/server1.pdf}
    \includegraphics<2>[width=12cm]{figure/server2.pdf}
    \includegraphics<3>[width=12cm]{figure/server3.pdf}
    \includegraphics<4>[width=12cm]{figure/server4.pdf}
  \end{figure}

  \begin{block}{問題}
    \uncover<4->{
    Web サービスを運用している会社では,運用しているサービスの応答が遅いと,顧客離れやクレームの被害を受けることがある.そのため,計算サーバーの応答の早さは重要である.
    }
  \end{block}

\end{frame}
%図を入れる

\begin{frame}{研究背景:スケジューリング問題との対応}
  \begin{block}{研究背景:スケジューリング問題との対応}
    Web サービスの問題を解決する方法の 1 つとして,\alert{タスクが到着してから,処理されるまでの時間を短くする}方法が挙げられる.計算サーバーにおける各時間は以下に対応する.つまり,計算サーバーへのタスク割り当ては,スケジューリング問題として捉えることができる.
    \begin{itemize}
      \item タスクの到着時刻とは,\alert{処理開始可能時刻}に対応する.
      \item タスクが到着してから,タスクの処理が開始されるまでの時間とは,\alert{実行開始待ち時間}に対応する.
    \end{itemize}
    \alert{最大実行開始待ち時間最小化問題(SPWT)}は,処理開始可能時刻を制約とし,最大の実行開始待ち時間の最小化を目的とするスケジューリング問題として捉えることができる.
  \end{block}

  \begin{block}{研究目的}
    機械モデルおよび機械数に着目し,問題の難しさに影響を与える特徴を明らかにする.
    また,SPWT の計算複雑さに基づいて,解法の提案を行う.
  \end{block}

\end{frame}
%%%%%%%%%%%%%%%%%%%%%%%%%%%%%%%%%%%%%%%%%%%%%%%%%%%%%%%%%%%%%%%%%%%
% \section{研究目的}
% \begin{frame}{研究目的}
%   \begin{block}{目的 1:SPWT の計算複雑さを明らかにする.}
%     機械モデルおよび機械数に着目し,問題の難しさに影響を与える特徴を明らかにする.
%   \end{block}
%
%   \begin{block}{目的 2:SPWT に対する効率的解法の提案.}
%     SPWT の計算複雑さに基づいて,解法の提案を行う.また,計算サーバーにおける解法の実験的評価を行う.
%   \end{block}
%
% \end{frame}
%%%%%%%%%%%%%%%%%%%%%%%%%%%%%%%%%%%%%%%%%%%%%%%%%%%%%%%%%%%%%%%%%%%
\section{定式化}

\subsection{最大実行開始待ち時間最小化問題の定式化}
\begin{frame}{定式化:最大実行開始待ち時間最小化問題}
  \begin{itemize}
    \item {無関連並列機械モデルにおける SPWT を決定問題として定義する.}
    \item {決定問題は,インスタンスと問題によって定義され,決定問題は判定とし
    て yes または no のいずれかを持つ.}
  \end{itemize}
  \begin{block}{入力}
    \begin{itemize}
      \item {ジョブの集合 $\mathcal{J} = \{J_1,J_2,\ldots,J_n\}$}
      \item {無関連機械の集合 $\mathcal{M} = \{M_1,M_2,\ldots,M_m\}$}
      \item {ジョブの処理開始可能時刻を返す関数 $r : \mathcal{J} \to \mathbb{N}$}
      \item {ジョブの処理時間を返す関数 $p : \mathcal{J} \times \mathcal{M} \to \mathbb{N}$}
      \item {実行開始待ち時間 $w$}
    \end{itemize}
  \end{block}
\end{frame}

\begin{frame}{定式化:最大実行開始待ち時間最小化問題}
  \begin{block}{問題}
    以下の条件を満たす $A : \mathcal{J} \to \mathcal{M}$ と $s : \mathcal{J} \to \mathbb{N}$ の対 $(A,s)$ が存在するか?
    \begin{itemize}
      \item {$\forall J \in \mathcal{J}\big[s(J) \ge r(J) \big]$}
      \begin{itemize}
        \item {各ジョブは処理開始可能時刻以降に処理を開始する.}
      \end{itemize}
      \item {$\forall J, J' \in \mathcal{J}\ \Big[ \big[J\ne J' \land A(J) = A(J')\big] \Rightarrow [s(J), s(J)+p(J,A(J))) \cap[s(J'), s(J')+p(J', A(J'))) = \emptyset \Big]$}
      \begin{itemize}
        \item {各機械は同時に複数のジョブを処理しない.}
        \item {各ジョブの処理を開始すると,完了するまで中断しない.}
      \end{itemize}
      \item {$\max\big\{s(J) - r(J) \mid J \in \mathcal{J}\big\} \le w$}
      \begin{itemize}
        \item {ジョブの処理開始可能時刻からその処理を開始するまでの待ち時間は $w$ 以下.}
      \end{itemize}
    \end{itemize}
  \end{block}
\end{frame}

\subsection{\textsc{3-SATISFIABILITY} の導入}
\begin{frame}{定式化:${\textcolor{white}{\mbox {\sc 3-SATISFIABILITY}}}$}
  \textsc{3-SATISFIABILITY (3-SAT) } は決定問題の 1 つで,この問題の計算複雑さは NP  完全であることが知られている.
  \begin{tabular}{cc}
    \begin{minipage}[]{0.7\hsize}
      \begin{block}{\textsc{3-SATISFIABILITY}}
        インスタンス:$(X,H)$
        \begin{itemize}
          \item $X=\{x_1,x_2,\dots ,x_n\}$
          \begin{itemize}
            \item $L_X = X \cup \{\bar x \mid x \in X\}$
          \end{itemize}
          \item $H \subseteq 2^{L_X}$ s.t. $\forall h \in H \big[|h| = 3\big]$
        \end{itemize}
        問題:以下を満たす $f : X \to \{0,1\}$ が存在するか?
        \begin{itemize}
          \item $\displaystyle \bigwedge_{h \in H} \bigg(\bigvee_{x \in h}f(x) \lor \bigvee_{\bar x \in h}\lnot f(x) \bigg) = 1$
        \end{itemize}
      \end{block}
    \end{minipage}
    \begin{minipage}[c]{0.3\hsize}
    \end{minipage}
    \vspace{3mm}
  \end{tabular}

  例えば,$X = \{x_1,x_2,x_3\}$,$H = \big\{ \{x_1, x_2, \bar x_3\}, \{\bar x_1, \bar x_2,\bar x_3\}\big\}$ のとき,
  $f(x_1) = 1, f(x_2) = 0, f(x_3) = 0$ が存在するので,判定は yesとなる.
\end{frame}
%%%%%%%%%%%%%%%%%%%%%%%%%%%%%%%%%%%%%%%%%%%%%%%%%%%%%%%%%%%%%%%%%%%
\section{多項式還元とは}
\begin{frame}{多項式還元とは:還元の流れ}
  \begin{tabular}{cc}
    \begin{minipage}[c]{0.5\hsize}
      \begin{block}{NP 完全}
        ある問題の計算複雑さが NP 完全であることを示すには,
        以下が成立することを示す.
        \begin{itemize}
          \item 問題がNPに属す.
          \item 問題がNP困難である.
        \end{itemize}
      \end{block}
    \end{minipage}
    \begin{minipage}[c]{0.5\hsize}
      \begin{textblock*}{0.4\linewidth}(210pt, 20pt)
        \centering
        \includegraphics[width=6cm,bb=0 0 350 260]{figure/reduction.pdf}
      \end{textblock*}
    \end{minipage}
  \end{tabular}
  \begin{block}{}
    本研究では以下が成立することを示す.
    \begin{enumerate}
      \item スケジュールにおける最大実行開始待ち時間が $w$ 以下であるかの判定が\alert{多項式時間}で可能である.
      \item \textsc{3-SAT} のインスタンスから SPWT のインスタンスが\alert{多項式時間}で構成可能であり,\textsc{3-SAT} の判定結果と構成したインスタンスに対する $w$ 以下となるスケジュールの存在判定結果が\alert{一致}する.
    \end{enumerate}
  \end{block}
\end{frame}
%%%%%%%%%%%%%%%%%%%%%%%%%%%%%%%%%%%%%%%%%%%%%%%%%%%%%%%%%%%%%%%%%%%
\section{問題の分析}
\begin{frame}{問題の分析:既存のスケジューリング問題との対応}
  \begin{block}{既存のスケジューリング問題との共通部分}
    \begin{itemize}
      \item {$w = 0$ のとき,処理開始可能時刻ちょうどで処理を開始しなければならない.}
      \begin{itemize}
        \item {\alert{JIT ジョブスケジューリング問題(SJIT)}と対応する.}
      \end{itemize}
      \item {SPWT において,納期 = (処理開始可能時刻 + 処理時間 + $w$ )と定義できる.}
      \begin{itemize}
        \item {\alert{処理開始可能時刻と納期を持つスケジューリング問題(SRTD)}と対応する.}
      \end{itemize}
    \end{itemize}
  \end{block}
  \begin{block}{既存のスケジューリング問題との差分}
    \begin{itemize}
      \item {SPWT は,SJIT の\alert{拡張問題}として捉えることができる.}
      \item {SPWT における納期は処理開始可能時刻に関連づけられる.}
      \begin{itemize}
        \item {SRTD の\alert{部分問題}として捉えることができる.}
      \end{itemize}
    \end{itemize}
  \end{block}
\end{frame}
%%%%%%%%%%%%%%%%%%%%%%%%%%%%%%%%%%%%%%%%%%%%%%%%%%%%%%%%%%%%%%%%%%%
\section{研究成果}
\begin{frame}{研究成果:問題の計算複雑さ}
  \begin{alertblock}{成果 1}
    無関連並列機械モデルにおいて,機械数が入力の一部の場合,SPWT が NP 完全であることを明らかにした.
  \end{alertblock}
\end{frame}

\begin{frame}{研究成果:ヒューリスティックの提案}
  \begin{alertblock}{成果 2}
    同一並列機械モデルにおける SPWT に対して,ヒューリスティックの開発と解法に対する実験的評価を行った.
  \end{alertblock}
  \begin{block}{貪欲アルゴリズムに基づいた解法}
    \begin{description}
      \setlength{\leftskip}{-10mm}
      \item[入力 :] $I = (\mathcal{J}, \mathcal{M},r,p,C)$
      \item[出力 :] スケジュールの集合 $\mathcal{S}$.
      \begin{description}
        \setlength{\leftskip}{-25mm}
        \item[Step 1.]
        $\mathcal{J}$ を処理開始可能時刻の昇順でソートする.
        \item[Step 2.]
        各 $1 \le i \le n$ における $J_i$ について以下の処理を繰り返す.
        \begin{description}
          \item[Step 2.1.]
          \setlength{\leftskip}{-40mm}
          最小完了時刻を持つ機械集合の要素 $M_a \in \left\{ \displaystyle \argmin_{M \in \mathcal{M}}C(\mathcal{J}_{S_M})\right\}$ を 1 つ求める.
          ここで,$S_{M_a} := S_{M_a} \cup J_i$ とする.ただし,$S_{M_a}$ は機械 $M_a \in \mathcal{M}$ におけるスケジュール.
        \end{description}
        \item[Step 3.]
        $\mathcal{S} = \{ S_M \mid M \in \mathcal{M}\}$ として,$\mathcal{S}$ を出力する.
      \end{description}
    \end{description}
  \end{block}
\end{frame}

\begin{frame}{研究成果:ヒューリスティックの実験的評価}

\end{frame}

\begin{frame}{研究成果:厳密解法の提案}
  \begin{alertblock}{成果 3}
    同一並列機械モデルにおける SPWT に対して,厳密解法の開発と計算時間の分析を行った.
  \end{alertblock}

\end{frame}
%%%%%%%%%%%%%%%%%%%%%%%%%%%%%%%%%%%%%%%%%%%%%%%%%%%%%%%%%%%%%%%%%%%
\section{今後の課題}
\begin{frame}{今後の課題}

\end{frame}
%%%%%%%%%%%%%%%%%%%%%%%%%%%%%%%%%%%%%%%%%%%%%%%%%%%%%%%%%%%%%%%%%%%
\end{document}
