\documentclass[oneside, 10pt, twocolumn]{jarticle}

\title{\bf{\rm
並列機械モデルにおける\\最大待ち時間最小化問題の計算論的分析}}

\author{宋研究室
\hspace{15pt}
天本 祐希 (15713004)}
\date{}

\usepackage{amsfonts}
\usepackage{setspace}
\setstretch{1.18} % ページ全体の行間を設定
\pagestyle{empty}
\oddsidemargin -5mm
\textwidth 170mm
\topmargin -28mm
\textheight 270mm
\columnsep 5mm

% sectionの大きさ
\makeatletter
\def\section{\@startsection {section}{1}{\z@}{-3.5ex plus -1ex minus
-.2ex}{2.3 ex plus .2ex}{\large\bf}}
% sectionの行間
\renewcommand{\section}{
\@startsection{section}{1}{\z@}
{.1\Cvs \@plus.0\Cdp \@minus.1\Cdp}%  上の空き
{.1\Cvs \@plus.1\Cdp \@minus.0\Cdp}%  下の空き
{\reset@font\large\bfseries}}      %  字の大きさ
\makeatother

\begin{document}
\maketitle
\thispagestyle{empty}
%%%%%%%%%%%%%%%%%%%%%%%%%%%%%%%%%%%%%%%%%%%%%%%%%%%%%%%%%%%%%%%%%%%%%%%%%%%%%%
\section{研究背景}
多くの高級自動車メーカーは,顧客の注文を受けた後に,製品の製造工場と製造開始日を決定し,それらの情報を顧客に伝える.
高級自動車メーカーは,生産拠点が少ないため,発注から製造を開始するまでに日数がかかる場合がある.製造開始までの待ち時間によっては,顧客からのクレームや注文のキャンセルを受けることがある.
このような問題を解決するために,各製品の製造開始までの待ち時間を短縮する必要がある.そのために,製品をどの工場でいつ製造を開始するかを考える.

上記の課題を解決することは,{\bf タスク}をどの{\bf 資源}で,{\bf いつ}処理を開始するかの決定を意味する.注文を受けた日を{\bf 処理開始可能時刻},注文を受けてから製造開始日までの期間を{\bf 待ち時間}としたとき,上記の問題は,{\bf 最大待ち時間最小化問題 (SWT) }として捉えることができる.SWT は,処理開始可能時刻を制約とし,最大待ち時間 (\mbox{\boldmath $W_{\max}$}) の最小化を目的とするスケジューリング問題である.
$W_{\max}$ を目的関数としたスケジューリング問題は,どの機械モデルにおいても未解決問題である.

\section{研究目的}
SWT の関連研究として,処理開始可能時刻付き最大遅れ時間最小化問題 (SRTD) や JIT ジョブ荷重和最大化問題 (SJIT) などがある.
SRTD は単一機械モデルにおいて,SJIT は無関連並列機械モデルにおいて機械数が入力の一部の場合,それぞれ強 NP 困難であることが示されている\cite{SRTD}\cite{SJIT}.本研究では,SWT に対して,以下を目的とする.
\begin{description}
  \item[目的 1 :]
  \underline{SWT の計算複雑さを明らかにする.}
\end{description}
SWT を決定問題として定義し,機械モデルおよび機械数に着目することで,どのような特徴が問題の難しさに影響を与えるかを明らかにする.

\begin{description}
  \item[目的 2 :]
  \underline{SWT に対する効率的解法の提案.}
\end{description}
SWT の計算複雑さに基づいて,解法の提案を行う.また,計算機を用いて解法の実験的評価を行う.

\section{研究成果}
本研究では,SWT に関して,以下の成果を得た.
\begin{description}
  \item[成果 1 : ]
  無関連並列機械モデルにおいて機械数が入力の一部の場合,\underline{SWT の NP 完全性}を示した.
\end{description}
SJIT における\textsc{3-SAT} からの還元手法に着目し,SWT の NP 完全性を示した.

\begin{description}
  \item[成果 2 : ]
  同一並列機械モデルにおける SWT に対し,\underline{ヒューリスティックを提案}した.
\end{description}
貪欲的解法に基づいたヒューリスティックを開発した.ヒューリスティックによる解から得られた最大待ち時間を $W_h$,最適解から得られた最大待ち時間を $W_{opt}$ としたとき,\mbox{\boldmath $\max\big\{W_h/W_{opt}\big\} = ???$} の結果が得られた.

\begin{description}
  \item[成果 3 : ]
  以下の改良を加え,同一並列機械モデルにおける SWT に対し,分割生成アルゴリズムおよび分枝限定法に基づいて,\underline{厳密解法を提案}した.
  \begin{itemize}
    \setlength{\leftskip}{-10mm}
    \item 分割生成アルゴリズムに対し,{\bf 分割の要素数 = 機械数}となる改良を加えた.
    \item 分枝限定法に対し,SRTD の部分問題に対する多項式アルゴリズムの概念を導入した.
  \end{itemize}
\end{description}
分割生成アルゴリズムの改良により,考慮する分割の数を減らすことで計算効率を向上させた.また,分枝限定法の改良により,列挙する実行可能解を減らし,計算効率を向上させた.
その結果,同じインスタンスに対して,計算時間を約 {\bf ??? 倍}にすることに成功した.

\begin{thebibliography}{9} %参考文献{載せる参考文献の数の上限}
  \bibitem{SRTD} % SRTDの計算複雑さの証明
  Garey, Johnson.
  Computers and Intractability A Guide to the Theory of NP-Completeness.
  W. H. Freeman And Co, pp. 13-244, 1990.
  \bibitem{SJIT} % SJITの計算複雑さの証明
  Sung, Vlach.
  Maximizing Weighted Number of Just-In-Time Jobs on Unrelated Parallel Machines. J SCHED 8, pp. 453-460, 2005.
  \vspace{-2mm}
\end{thebibliography}

\end{document}
