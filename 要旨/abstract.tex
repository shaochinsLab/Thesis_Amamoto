\documentclass[oneside, 10pt, twocolumn]{jarticle}

\title{\bf{\rm
並列機械モデルにおける\\最大待ち時間最小化問題の計算論的分析}}

\author{宋研究室
\hspace{15pt}
天本 祐希 (15713004)}
\date{}

\usepackage{amsfonts}
\usepackage{setspace}
\setstretch{1.18} % ページ全体の行間を設定
\pagestyle{empty}
\oddsidemargin -5mm
\textwidth 170mm
\topmargin -28mm
\textheight 270mm
\columnsep 5mm

% sectionの大きさ
\makeatletter
\def\section{\@startsection {section}{1}{\z@}{-3.5ex plus -1ex minus
-.2ex}{2.3 ex plus .2ex}{\large\bf}}
% sectionの行間
\renewcommand{\section}{
\@startsection{section}{1}{\z@}
{.1\Cvs \@plus.0\Cdp \@minus.1\Cdp}%  上の空き
{.1\Cvs \@plus.1\Cdp \@minus.0\Cdp}%  下の空き
{\reset@font\large\bfseries}}      %  字の大きさ
\makeatother

\begin{document}
\maketitle
\thispagestyle{empty}
%%%%%%%%%%%%%%%%%%%%%%%%%%%%%%%%%%%%%%%%%%%%%%%%%%%%%%%%%%%%%%%%%%%%%%%%%%%%%%
\section{研究背景}
受注生産方式では,顧客注文を受けてから,その受注製品の生産を全体の生産計画に組み込むため,どの生産拠点でいつ製造開始するかを決定する.高級自動車メーカー,SIerなど,受注生産方式を採用しており,Web サービスなどタスク処理も受注生産方式とみなすことができる.
保有する生産拠点の数と受注状況によって,受注から製造開始までの待ち時間が長くなることがあり,顧客満足度の低下や注文のキャンセルなどに繋がる.
よって,製造開始までの{\bf 待ち時間}を短縮させるための生産計画を立てることは重要な課題である.

各注文をジョブに対応させ,受注日時を{\bf 処理開始可能時刻},製造期間を{\bf 処理時間},受注日時と製造期間の和を{\bf 納期}とすると,処理開始可能時刻付き最大遅れ時間最小化問題 (SRTD) における最大遅れ時間は最大待ち時間に対応し,JIT ジョブ荷重和最大化問題 (SJIT) において全てのジョブを JIT で処理することは最大待ち時間が 0 の場合に対応する. SRTD は単一機械モデルにおいて,SJIT は無関連並列機械モデルにおいて機械数が入力の一部の場合,それぞれ強 NP 困難であることが示されている \cite{SRTD}\cite{SJIT}.
しかし,直接的に目的関数を待ち時間とするスケジューリング問題を扱う従来研究は,調査した限り存在しない.

\section{研究目的}
上記の問題を{\bf 最大待ち時間最小化問題 (SWT) }として定式化した.SWT の拡張問題である SRTD が単一機械モデルにおいて強 NP  困難であるため,SWT も NP 困難であることが予想されるが,明らかではない.
\begin{description}
  \item[目的 1 :]
  \underline{SWT の計算複雑さを明らかにする.}
\end{description}
機械モデルおよび機械数に着目することで,どのような特徴が問題の難しさに影響を与えるかを明らかにする.

\begin{description}
  \item[目的 2 :]
  \underline{SWT に対する効率的解法の提案.}
\end{description}
SWT の計算複雑さに基づいて,解法の提案を行う.また,計算機を用いて解法の実験的評価を行う.

\section{研究成果}
%本研究では SWT に対して以下の成果を得た.
\begin{description}
  \item[成果 1 : ]
  無関連並列機械モデルにおいて機械数が入力の一部の場合,\underline{SWT が NP 困難}であることを明らかにした.
\end{description}
SWT を決定問題として定義した問題の NP 完全性を示し,SWT が NP 困難であることを明らかにした.

\begin{description}
  \item[成果 2 : ]
  同一並列機械モデルにおける SWT に対し,分割生成アルゴリズムおよび分枝限定法に以下の改良を加え,\underline{厳密解法を提案}した.
  \begin{description}
    \setlength{\leftskip}{-7mm}
    \item[2.1]
    分割生成アルゴリズムに対し,{\bf 分割の要素数 = 機械数}となる改良を加えた.
    \item[2.2]
    分枝限定法に対し,SRTD の部分問題に対する多項式アルゴリズムの概念を導入した.
    \item[2.3]
    生成した分割の各要素におけるコストの下限の降順で,分枝限定法による順列生成を行う.
  \end{description}
\end{description}
成果 2.1 より,考慮する分割の数を減らし,成果 2.2,成果 2.3 より,列挙する実行可能解を減らした.以上の改良により,同じインスタンスに対して計算時間を最大 {\bf 90 %}削減した.

\begin{description}
  \item[成果 3 : ]
  同一並列機械モデルにおける SWT に対し,\underline{ヒューリスティックを提案}した.
\end{description}
貪欲アルゴリズムに基づいたヒューリスティックを開発した.ヒューリスティックにより出力された解から得られた最大待ち時間を $W_h$,最適解から得られた最大待ち時間を $W_{opt}$ としたとき,\mbox{\boldmath $\max\big\{W_h/W_{opt}\big\} = 24$} の結果が得られた.

\begin{thebibliography}{9} %参考文献{載せる参考文献の数の上限}
  \bibitem{SRTD} % SRTDの計算複雑さの証明
  Garey, Johnson.
  Computers and Intractability A Guide to the Theory of NP-Completeness.
  W. H. Freeman And Co, pp. 236-244, 1990.
  \vspace{-2mm}
  \bibitem{SJIT} % SJITの計算複雑さの証明
  Sung, Vlach.
  Maximizing Weighted Number of Just-In-Time Jobs on Unrelated Parallel Machines. J SCHED 8, pp. 453-460, 2005.
\end{thebibliography}

\end{document}
