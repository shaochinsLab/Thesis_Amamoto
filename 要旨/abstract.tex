\documentclass[oneside, 10pt, twocolumn]{jarticle}

\title{\bf{\rm
並列機械モデルにおける\\最大待ち時間最小化問題の計算論的分析}}

\author{宋研究室
\hspace{15pt}
天本 祐希 (15713004)}
\date{}

\usepackage{amsfonts}
\usepackage{setspace}
\setstretch{1.18} % ページ全体の行間を設定
\pagestyle{empty}
\oddsidemargin -5mm
\textwidth 170mm
\topmargin -28mm
\textheight 270mm
\columnsep 5mm

% sectionの大きさ
\makeatletter
\def\section{\@startsection {section}{1}{\z@}{-3.5ex plus -1ex minus
-.2ex}{2.3 ex plus .2ex}{\large\bf}}
% sectionの行間
\renewcommand{\section}{
\@startsection{section}{1}{\z@}
{.1\Cvs \@plus.0\Cdp \@minus.1\Cdp}%  上の空き
{.1\Cvs \@plus.1\Cdp \@minus.0\Cdp}%  下の空き
{\reset@font\large\bfseries}}      %  字の大きさ
\makeatother

\begin{document}
\maketitle
\thispagestyle{empty}
%%%%%%%%%%%%%%%%%%%%%%%%%%%%%%%%%%%%%%%%%%%%%%%%%%%%%%%%%%%%%%%%%%%%%%%%%%%%%%
\section{研究背景}
受注生産方式では,顧客注文を受けてから,その受注製品の生産を全体の生産計画に組み込むため,どの生産拠点でいつ製造開始するかを決定する.高級自動車メーカー,***など,受注生産方式を採用しており,Web サービスなどタスク処理も受注生産方式とみなすことができる.
保有する生産拠点の数と受注状況によって,受注から製造開始までの待ち時間が長くなることがあり,顧客満足度の低下や注文のキャンセルなどに繋がる.
よって,製造開始までの{\bf 待ち時間}を短縮させるための生産計画を立てることは重要な課題である.

製造開始までの待ち時間を短縮する問題の関連したスケジューリング問題が存在する.
処理開始可能時刻付き最大遅れ時間最小化問題 (SRTD) は,最大納期遅れ時間の最小化を目的とするスケジューリング問題である.SRTD における納期を処理開始可能時刻 + 処理時間としたとき,納期遅れ時間は待ち時間に対応する.
JIT ジョブ荷重和最大化問題 (SJIT)は,JIT ジョブの荷重和の最大化を目的とするスケジューリング問題である.受注から製造開始までの待ち時間を 0 としたとき,すべての受注製品を JIT ジョブとして捉えることができる.しかし,目的関数を待ち時間とするスケジューリング問題は,従来研究の調査の結果発見されていない.

\section{研究目的}
受注日を{\bf 処理開始可能時刻},受注から製造開始までの期間を{\bf 待ち時間}としたとき,待ち時間を短縮する問題は,{\bf 最大待ち時間最小化問題 (SWT) }として捉えることができる.
% SWT は,処理開始可能時刻を制約とし,最大待ち時間 (\mbox{\boldmath $W_{\max}$}) の最小化を目的とするスケジューリング問題である.
本研究では,SWT に対して以下を目的とする.
\begin{description}
  \item[目的 1 :]
  \underline{SWT の計算複雑さを明らかにする.}
\end{description}
SWT を決定問題として定義し,機械モデルおよび機械数に着目することで,どのような特徴が問題の難しさに影響を与えるかを明らかにする.

\begin{description}
  \item[目的 2 :]
  \underline{SWT に対する効率的解法の提案.}
\end{description}
SWT の計算複雑さに基づいて,解法の提案を行う.また,計算機を用いて解法の実験的評価を行う.

\section{研究成果}
本研究では,SWT に関して,以下の成果を得た.
\begin{description}
  \item[成果 1 : ]
  無関連並列機械モデルにおいて機械数が入力の一部の場合,\underline{SWT の NP 完全性}を示した.
\end{description}
SJIT における\textsc{3-SAT} からの還元手法に着目し,SWT の NP 完全性を示した.

\begin{description}
  \item[成果 2 : ]
  以下の改良を加え,同一並列機械モデルにおける SWT に対し,分割生成アルゴリズムおよび分枝限定法に基づいて,\underline{厳密解法を提案}した.
  \begin{itemize}
    \setlength{\leftskip}{-10mm}
    \item 分割生成アルゴリズムに対し,{\bf 分割の要素数 = 機械数}となる改良を加えた.
    \item 分枝限定法に対し,SRTD の部分問題に対する多項式アルゴリズムの概念を導入した.
  \end{itemize}
\end{description}
分割生成アルゴリズムの改良により,考慮する分割の数を減らすことで計算効率を向上させた.また,分枝限定法の改良により,列挙する実行可能解を減らし,計算効率を向上させた.
その結果,同じインスタンスに対して,計算時間を約 {\bf ??? 倍}にすることに成功した.

\begin{description}
  \item[成果 3 : ]
  同一並列機械モデルにおける SWT に対し,\underline{ヒューリスティックを提案}した.
\end{description}
貪欲的解法に基づいたヒューリスティックを開発した.ヒューリスティックによる解から得られた最大待ち時間を $W_h$,最適解から得られた最大待ち時間を $W_{opt}$ としたとき,\mbox{\boldmath $\max\big\{W_h/W_{opt}\big\} = ???$} の結果が得られた.

\begin{thebibliography}{9} %参考文献{載せる参考文献の数の上限}
  \bibitem{SRTD} % SRTDの計算複雑さの証明
  Garey, Johnson.
  Computers and Intractability A Guide to the Theory of NP-Completeness.
  W. H. Freeman And Co, pp. 13-244, 1990.
  \bibitem{SJIT} % SJITの計算複雑さの証明
  Sung, Vlach.
  Maximizing Weighted Number of Just-In-Time Jobs on Unrelated Parallel Machines. J SCHED 8, pp. 453-460, 2005.
  \vspace{-2mm}
\end{thebibliography}

\end{document}
